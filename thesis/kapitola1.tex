\chapter{Základné pojmy a definície}

\label{kap:zakladne} % id kapitoly pre prikaz ref

\begin{description}
\item [Latencia] - reakčný čas označújúci dobu medzi akciou a následnou reakciou. Vo webových 
technológiách sa pod letenciou rozumie čas, medzi poslaním webovej požiadavky a následnej
odpovede zo servera.
\item [Dokument] - zdroj, ktorý sa zdieľa medzi jedotlivými používateľmi
\item [Replika] - inštancia dokumentu, ktorá je zdieľaná medzi viacerými používateľmi. Keď nejaký
používateľ zmení repliku, tak sa dané zmeny prejavia aj v ostatných replikách u iných
používateľov.
\item [Úplne usporiadaná množina] - \textit{taktiež známa pod názvom lineárne usporiadaná množina.} 
Úplne usporiadaná množina je usporiadaná množina v ktorej sú každé dva rôzne prvky porovnateľné. 
Takouto reláciou je napríklad množina prirodzených čísel usporiadaná reláciou "menší ako".
\item [Hustý priestor] - vzťah na úplne usporiadaných množinách M, ktorý zabezpečuje, že pre 
ľubovolné $p, q \in M$ musí existovať nejaký prvok $r$, pričom platí $p \leq r$ a 
$r \leq q$.
\item [Jednostránková webová aplikácia] - typ webovej stránky, ktorý načíta zo servera jednu HTML
stránku a pri interakcii so stránkou sa jej obsah mení dynamicky vo webovom prehliadači. Takéto
stránky predstavujú moderný trend oproti klasickým statickým stránkam, ktoré pri interakciách
žiadali server o nový obsah stránky, čo spôsobovalo dlhšie zobrazovanie.
\end{description}
