\chapter{Základné pojmy a definície}

\label{kap:zakladne} % id kapitoly pre prikaz ref

\begin{description}
\item [Latencia] - reakčný čas označújúci dobu medzi akciou a následnou reakciou. Vo webových 
technológiách sa pod letenciou rozumie čas, medzi poslaním webovej požiadavky a následnej
odpovede zo servera.
\item [Dokument] - zdroj, ktorý sa zdieľa medzi jedotlivými používateľmi
\item [Replika] - inštancia dokumentu, ktorá je zdieľaná medzi viacerými používateľmi. Keď nejaký
používateľ zmení repliku, tak sa dané zmeny prejavia aj v ostatných replikách u iných
používateľov.
\item [Klient] - klient predstavuje časť programu, ktorá je prístupná používateľovi. V tejto práci
je klient jednostránková webová aplikácia.
\item [Úplne usporiadaná množina] - \textit{taktiež známa pod názvom lineárne usporiadaná množina.} 
Úplne usporiadaná množina je usporiadaná množina v ktorej sú každé dva rôzne prvky porovnateľné. 
Takouto reláciou je napríklad množina prirodzených čísel usporiadaná reláciou "menší ako".
\item [Hustý priestor] - vzťah na úplne usporiadaných množinách M, ktorý zabezpečuje, že pre 
ľubovolné $p, q \in M$ musí existovať nejaký prvok $r$, pričom platí $p \leq r$ a 
$r \leq q$.
\item [Jednostránková webová aplikácia] - typ webovej stránky, ktorý načíta zo servera jednu HTML
stránku a pri interakcii so stránkou sa jej obsah mení dynamicky vo webovom prehliadači. Takéto
stránky predstavujú moderný trend oproti klasickým statickým stránkam, ktoré pri interakciách
žiadali server o nový obsah stránky, čo spôsobovalo dlhšie zobrazovanie.
\item [Idempotencia] - je vlastnosť algebraických operácií. Operácia je idempotentná, ak jej
opakovaným použitím na nejaký vstup vznikne rovnaký výstup, ako vznikne jediným použitím danej
operácie.
\item [Optimistické zobrazovanie] - využíva sa v prípade, že aplikácia zobrazuje nejaké entity a
zároveň podporuje vytváranie nových. V optimistickom zobrazovaní užívateľské rozhranie predpokladá,
že odoslaná požiadavka na server prebehne úspešne. Dáta sú preto zobrazované hneď pri odoslaní
požiadavky na server. To umožňuje vnímavejší zážitok používateľa, pretože nemusí čakať, kým sa
zmeny prejavia aj na serveri.
\item [NoSQL databáza] - je databázový koncept, v ktorom dátové úložisko a spracovanie dát využíva
iné prostriedky ako tabuľkové schémy tradičnej relačnej databázy. Databázy bez SQL sú často vysoko
optimalizované pre typ klúč-hodnota.
\end{description}
