\chapter{Základné pojmy a definície}

\label{kap:zakladne} % id kapitoly pre prikaz ref

\begin{itemize}
\item Latencia - reakčný čas označújúci dobu medzi akciou a následnou reakciou. Vo webových 
technológiách sa pod letenciou rozumie čas, medzi poslaním webového requestu a následnej
odpovede zo servera.
\item Dokument - zdroj, ktorý sa zdieľa medzi jedotlivými používateľmi
\item Replika - inštancia dokumentu, ktorá je zdieľaná medzi viacerými používateľmi. Keď nejaký
používateľ zmení repliku, tak sa dané zmeny majú prejaviť aj na ostatných replikách u iných
používateľov.
\item Úplne usporiadaná množina - \textit{taktiež známa pod názvom lineárne usporiadaná množina.} 
Úplne usporiadaná množina je usporiadaná množina v ktorej sú každé dva rôzne prvky porovnateľné. 
Takouto reláciou je napríklad množina prirodzených čísel usporiadaná reláciou "menší než".
\item Hustý priestor - vzťah na úplne usporiadaných množinách M, ktorý zabezpečuje, že pre 
ľubovolné $p, q \in M$ musí existovať nejaký prvok $r$, pričom platí $p \leq r$ a 
$r \leq q$.
\end{itemize}
