\chapter{Návrh programu}

\label{kap:design} % id kapitoly pre prikaz ref

Program sa skladá z viacerých častí, ktorých návrhy popíšeme v tejto kapitole.
Návrh je založený na klient-server architektúre. Program sa preto dá rozdeliť na 
podkapitoly \textit{klient} a \textit{server}.

% TODO: obrazok popisujuci entity

\section{Klient}
Klient je v našej práci webová aplikácia cez ktorú komunikuje so serverom. 
V návrhu klienta sa snažíme hlavne vytvoriť jednoduchý UX/UI, ktorý je intuitívny na
použitie. Kedže aplikácia má byť používaná ako editor, je nutné aby sa jednotlivé časti pohodlne 
čítali a písmo bolo dostatočne veľké. Naopak, nepredpokladáme použitie editora na mobilných
zariadeniach. Preto je dizajn klienta tvorený pre použitie na počítačoch a na veľkých obrazovkách.
Klientská aplikácia obsahuje:
\begin{itemize}
\item prihlasovanie a registrácia
% TODO: chceme rozhranie pre bezneho pozuivatela kde si vyberie skupinu a atd..
\item panel so súbormi
\item administrátorské rozhranie
\item zdieľateľný editor
\item ovládací panel
\end{itemize}

\subsection{Prihlasovanie a registrácia}
Prihlasovanie sa slúži na jednoduché kategorizovanie používateľova na prístup ke editoru. 
Používateľov rozdeľujeme na bežných používateľov a administrátorov. Bežný používatelia predstavujú
zovšeobecnený pojem pre študentov, ktorý vedia byť časťou skupín (kurzov) v ktorých majú prístup
iba k úlohám prístupným pre danú skupinu. O tom do akej skupiny aký bežný používateľ patrí, 
rozhoduje administrátor. Prihlasovanie by však malo byť pre obe typy používateľov rovnaké.

Registrácia umožňuje vytvoriť nového používateľa. Zaujímavá otázka je, či umožniť vytváranie
administrátorov priamo pri regsitrácii, alebo nejakým iným spôsobom. Rozhodol som sa, že je lepšie
ak používateľ nemusí pri registrácii riešiť koncept administrátorov, a teda umožniť iba vytváranie
bežných používateľov. V aplikácii však umožňujem povýšenie bežného používateľa na administrátora.

\subsection{Panel so súbormi}
Úlohy sa môžu skladať z viacero súborov. Napríklad môže existovať súbor, ktorý
obsahuje zadanie a súbor (prípadne viac súborov) na zdrojový kód. Panel so súbormi dovoľuje
používateľovi pohodlne preklikávať a zvoliť aktívny súbor, ktorý sa zobrazuje v editore.

\subsection{Administrátorské rozhranie}
Administrátor má plný prístup ku všetkým informáciám na serveri. Po prihlásení sa administrátorovi
automaticky zobrazí administrátorské rozhranie, ktoré je úplne iné ako rozhranie pre bežného
používateľa. Adminstrátorské rozhranie umožňuje:
\begin{itemize}
\item zobraziť informácie o zaregistrovaných používaťeloch
\item zobraziť všetky odovzdané zadania
\item zobraziť všetky uložené zadania
\item vytvoriť nové zadanie
\item upraviť skupiny pre jednotlivé zadania a používaťelov
\end{itemize}

\subsection{Zdieľateľný editor}
Najzaujímavejšou častou práce je kolaboratívny editor umožnujúci bežným používateľom naraz
upravovať kód. V editore musí byť jasné, ktorý súbor sa práve upravuje a bolo by
tiež fajn vedieť nejaké informácie o vzdialených používaťeloch ako napríklad prihlasovacie meno, 
pozícia kurzora...

\subsection{Ovládací panel}
Samotná úprava kódu je dôležitá, ale okrem toho je potrebné verzionovanie
súborov, teda nejaké ukladanie a načítavanie. Okrem toho by používateľ mal mať možnost práve
upravovaný kód spúštať na vlastnom vstupe alebo na skrytých vstupoch danej úlohy.


\section{Server}
Server slúži na komunikáciu s klientom a obsahuje všetky potrebné informácie o používateľoch,
administrátoroch, skupinách a vytvorených zadaniach.

Server má viacero funkcií:
\begin{itemize}
\item synchronizácia replík klientov
\item uchovávať informácie o používateľoch a zadaniach
\item poskytovať API pre klienta
\item zbiehanie kódu, poskytovanie odpovede klientom
\end{itemize}

\subsection{Synchronozácia replík klientov}
Najdôležitejšou funckiou servera je synchronizácia replík klientov ak nastane nejaká zmena
dokumentu. Server na toto nepotrebuje veľkú logiku, ale potrebuje korektne preposielať informácie
medzi klientami a reagovať na prípadne komunikačné chyby (odoslať správu znovu).

\subsection{Uchovávať informácie o používateľoch a zadaniach}
Po registrácii sa musí na serveri uložiť používateľ aby sa následne mohol opätovne prihlasovať.
Okrem toho klient môže ukladať a načítavať zadania, ktoré tiež musia byť nejakým spôsobom na
serveri uložené.

\subsection{Poskytovať API pre klienta}
Hlavnou funckiou servera je poskytovať API rozhranie, ktoré môže klient zavolať a vykonať nejakú
akciu na serveri. Časť rozhrania služi na získanie informácii zo servera a čast vytvára
na serveri nové položky (používatelia, zadania).

\subsection{Zbiehanie kódu, poskytovanie odpovede klientom}
Kód, ktorí klienti napíšu sa má dať spustiť v izolovanom prostredí servera. Aplikácia by mala
poskytnúť dve možnosti ako testovať klientský kód:
\begin{itemize}
\item na vlastnom vstupe
\item na skrytých vstupoch
\end{itemize}

Pri testovaní na vlastnom vstupe by klient mal mať možnosť zadať vstup a potom odoslať požiadavku
na testovanie na server. Po otestovaní by sa mal klientovi zobraziť výsledok testovania. Testovanie
na vlastných vstupoch nemusí byť ukladané na serveri, stačí ak server na požiadavku odpovie.
Ak by výsledok testovania skončil chybou pri kompilácii, prípadne chybou počas behu programu,
daná chybová hláška sa má zobraziť ako výstup programu.

Okrem testovania na vlastných vstupoch si vie klient otestovať kód aj na skrytých vstupoch a
výstupoch zadania. Takéto skryté vstupy vie pridať iba administrátor a nikdy by nemali byť zobrazené
bežnému používateľovi. Každé testovanie na skrytých vstupoch musí byť uložené na serveri, aby si ho
administrátor vedel pozrieť keď bude ohodnocovať dané zadanie bežného používateľa. Po odoslaní
požiadavky na server na otestovanie na skrytých vstupoch by sa mali zobraziť výsledky testovania.
Skrytých vstupov môže byť viac, ale testuje sa na vstupoch zaradom a keď sa nájde vstup, na ktorom
program dáva nesprávny výstup tak sa v testovaní ďalej nepokračuje.
Výsledok testovania jedného vstupu môže byť:
\begin{itemize}
\item OK - Kompilácia a aj beh programu prebehli v poriadku a výsledok programu na danom vstupe
je totožný so správnym výstupom
\item WA - Kompilácia prebehla v poriadku a program úspešne skončil, ale výstup sa nezhoduje so
správnym výstupom
\item TLE - Kompilácia prebehla v poriadku ale program prekročil časový limit, ktorý je dostupný 
pre dané zadanie.
\item RTE - Kompilácia prebehla v poriadku, ale program počas behu programu spadol.
\item CE - Kompilácia programu sa nepodarila.
\end{itemize}
\textit{(V prípade, že jazyk, v ktorom je kód napísaný je interpretovaný, sa kompilácia preskakuje.
Stále však môžu nastať všetky výsledky testovania okrem CE)}
