\chapter{Funkcia servera}

\label{kap:server} % id kapitoly pre prikaz ref

Server má viacero funkcií:
\begin{itemize}
\item synchronizácia klientov
\item prihlasovanie pre administrátorov
\item ukladanie skrytých testov
\item zbiehanie kódu, poskytovanie odpovede klientom
\end{itemize}


\subsection{Synchronozácia klientov}
Najdôležitejšou funckiou servera je synchronizácia všetkých klientov ak nastane nejaká zmena
dokumentu. Server na toto nepotrebuje veľkú logiku, ale potrebuje korektne preposielať informácie
medzi klientami a reagovať na prípadne komunikačné chyby (vačšinou odosľat správu znovu).


\subsection{Prihlasovanie pre administrátorov}
Súčasťou práce je aj administrátorské rozhranie, do ktorého sa dá prihlásiť a vytvárať v ňom
skryté testy.

\subsection{Prihlasovanie pre administrátorov}
Administrátori vedia pridávať skryté testy, na ktorých
sa dá spúštať klientský kód. Toto funguje na jednoduchom princípe skupín. Administrátor vytvorí 
skupinu, do ktorej sa klienti vedia prihlásiť. Ak je klient súčasťou nejakej skupiny, tak vie svoj
kód zbiehať voči testom vytvorených administrátorom.

\subsection{Zbiehanie kódu, poskytovanie odpovede klientom}
Ďalšou dôležitou častou je možnosť zbehnúť kód na serveri. Kód sa dá zbehnúť viacerými spôsobmi:
\begin{itemize}
\item na vlastných testoch
\item na skrytých testoch (iba v prípadne ak je klient členom nejakej skupiny)
\item na vlastnom vstupe
\end{itemize}
