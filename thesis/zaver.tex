\chapter*{Záver}  % chapter* je necislovana kapitola
\addcontentsline{toc}{chapter}{Záver} % rucne pridanie do obsahu
\markboth{Záver}{Záver} % vyriesenie hlaviciek

Podarilo sa nám implementovať celý návrh programu, ktorý je pripravený uľahčiť výučbu programovania
ako aj vytváranie jednoduchých zadaní pre študentov. Implementáciu predchádzalo študijné obdobie
zaoberajúce sa fungovaním zdieľateľnosti editorov a izolovaným spúštaním kódu. V samotnej
implementácii sme použili viaceré moderné knižnice, ktoré nám uľahčili vývoj programu. Súčasný stav
aplikácie je použiteľný ale len pre testovacie úcely. Aplikácia nijako nerieši bezpečnosť a študenti
môžu veľmi ľahko získať prístup k dátam, ktoré by mali byť dostupné iba pre administrátora. Okrem
toho spôsob testovania zdrojových kódov nezabezpečuje prostredie dokonale. Najlepšie by bolo, ak by
sa kód testoval na testovači v skutočnom virtuálnom prostredí, čo doker neposkytuje.

Zdrojový kód aplikácie je spravovaný prostredníctvom služby GitHub, ktorá umožňuje spoluprácu pri
vytváraní voľne šíriteľných programov. Pri implementácii boli dodržiavané štandardy programovania v
jazyku TypeScript, SQL a knižnice React. Okrem toho je zdrojový kód písaný v anglickom jazyku, takže
na knižnici môžu pracovať aj študenti z iných krajín. Knižnica je teda pripravená na prípadný budúci
rozvoj.
