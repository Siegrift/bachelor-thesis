\chapter{Využiteľnost v praxi a podobné programy}

\label{kap:usability_and_similar} % id kapitoly pre prikaz ref

\section{Využiteľnost v praxi}
Myšlienka kolaboratívnych editorov a všeobecne kolaboratívnych prostredí sa dá v praxi využiť vo 
viacerých oblastiach. Už v roku 2012, bola veľkosť latencie veľmi blízko teoretickému limitu,
ktorý vieme dosiahnuť \cite{latency_bottleneck}. Stále sa však zlepšuje prístupnosť internetu
ľuďom. V súčasnej dobe je bežné, že ľudia pracujúci za počítačom robia veci z pohodlia domova.
Kolaboratívne programy umožnujú prepojenie týchto vzdialených používateľov a umožnuje im zdieľanie
a upravovanie dokumentov.

Kolaboratívne editory sú populárne na účel vzdialených pohovorov, kde pohovorujúci komunikuje s
pohovorovaným cez internet alebo telefón a ten môže programovať do editora, pričom pohovorujúci 
vidí zmeny okamžite. Veľkou výhodou je, že pohovorovaný nemusí cestovať do kancelárii firmy, u
ktorej vykonáva pohovor.

Ďalším využitím je párové programovanie. Striktná definícia párového programovania umožnuje
písať naraz iba jednému, zatiaľ čo druhý je len pozorovateľ. Častokrát je však výhodné mať možnosť
zasiahnuť do kódu aj ako pozorovateľ, prípadne programovať naraz.

Ďalším veľkým využitím, sú spoločné projekty ako také. \textit{Pod pojmom "projekt" máme na mysli
ľubovolnú prácu alebo zadanie, ktoré vykonáva viacero ľudí}. Môže ísť o projekt vo firme, školské
zadanie... Častokrát sa tieto projekty dajú rozdeliť na zmysluplné časti, ktoré vedia byť spravené 
nezávisle od seba.

Posledným využitím a zároveň aj cieľom tejto práce bolo vytvoriť učebné prostredie, ktoré by sa na
univerzite dalo používať na informatických predmetoch. Učiteľ by zadal úlohy, ktoré by študenti
vedeli riešiť, pričom by mali možnosť pracovať na zadaní súčasne.

\section{Porovnanie s existujúcimi programami}
Ako sme už spomenuli v kapitole \ref{kap:zdialtelnost}, existuje viacero programov, ktoré podporujú
kolaboratívne upravovanie, pričom stále vznikajú nové. Sústreďujme sa ale iba na porovnanie s
programami, ktoré podporujú kolaboratívne úpravy \textbf{textového} dokumentu. Tieto programy sa
dajú rozdeliť do viacerých kategorii, ktoré vyjadrujú odlišnosť oproti návrhu tejto bakalárskej
práce. Programy, ktoré tu budú spomenuté sa odlišujú buď v použitých technológiách alebo tým, že nie
sú zamerané na progoramovanie zdrojových kódov (a teda nepodporujú ani spúštanie a testovanie)
prípadne podporujú iba nejaké jazyky.

\subsection{Využívajúce technológie dostupné iba na niektorých platformách}
Do tejto kategórie patria programy vytvorené pre operačné prostredie \textit{macOs}.
Príkladom takéhoto programu je \textit{SubEthaEdit}, ktorý začal revolúciu kolaboratívneho 
programovania. V súčastnosti je editor dostupný zadarmo a je open source. Ďalším rozdielom oproti
návrhu v bakalárskej práci je nemožnosť daný kód zbiehať.

\subsection{Používajúce inú technológiu konkurentých úprav}
Na spracovávanie konkurentých úprav poznáme dve dobre preskúmané technológie (\textit{OT} alebo
\textit{CRDT}). Vačšina produktov používa \textit{CRDT}, lebo je to výrazne ľahšie na implementáciu
a udržiavanie. Medzi programy využívajúce OT patria editory \textit{SubEthaEdit, ACE, Gobby,
MoonEdit}... Mnohé z nich prestali byť ďalej vyvíjané, alebo sa o ne stará open source komunita.

\subsection{Online dostupné textové editory}
Ďalšou kategóriou sú online editory, teda webové stránky. Niektoré su veľmi populárne
ale poskytujú iba čiastočnú funkcionalitu. Napríklad \textit{Ideone} je editor, v ktorom
sa dá zadarmo spúštať kód vo viac ako 50 jazykoch. Neumožňuje však kolaboratívne editovanie
a testovanie na skrytých vstupoch.
Jedným z najpoužívanejších voľne dostupných editorov je \textit{CodeShare}, ktorý umožnuje
kolaboratívne písanie, podporuje zvýrazňovanie syntaxe mnohých jazykov a obsahuje ďalšie 
dopľnujúce prvky ako napríklad videokamera pre vytvorenie editora spolu s videom.

Ako posledný si predstavme \textit{CodeSandbox}, ktorý podporuje iba písanie JavaScriptu, je 
však zaujímavý tým, že je open source a používa veľmi pokročilý editor VSCode, čo je ďalší open
source produkt. \textit{CodeSandbox} umožnuje aj kolaboratívne upravovanie. Je však zameraný iba
na JavaScript a je používaný hlavne na rýchle skúšanie webových frameworkov alebo krátkych
útržkov javascriptových kódov. Výhodou je, že používateľ nemusí nič inštalovať, lebo 
\textit{CodeSandbox} má všetko potrebné nainštalované na serveri. Okrem krátkych skúšaní projektov,
umožnuje aj pohodlnú integráciu s gitom, pridávanie ďalších knižníc, live reload...

\subsection{Platené programy}
Poslednou veľkou skupinou sú platené programy, ktoré poskytujú viac funkcionality ale ich použitie
je platené. Príkladmy týchto editorov sú \textit{CodeBunk} a \textit{CoderPad}. Mnohé z nich
poskytujú aj spúštanie kódu, prípadne videokameru.
